\documentclass[lang=cn,10pt]{elegantbook}

\title{ElegantBook:优美的 \LaTeX{} 书籍模板}
\subtitle{Elegant\LaTeX{} 经典之作}

\author{Ethan Deng \& Liam Huang}
\institute{Elegant\LaTeX{} Program}
\date{April 9, 2022}
\version{4.3}
\bioinfo{自定义}{信息}

\extrainfo{不要以为抹消过去,重新来过,即可发生什么改变。—— 比企谷八幡}

\setcounter{tocdepth}{3}

\logo{logo-blue.png}
\cover{cover.jpg}

% 本文档命令
\usepackage{array}
\newcommand{\ccr}[1]{\makecell{{\color{#1}\rule{1cm}{1cm}}}}

% 修改标题页的橙色带
% \definecolor{customcolor}{RGB}{32,178,170}
% \colorlet{coverlinecolor}{customcolor}

\begin{document}

\maketitle
\frontmatter

\tableofcontents

\mainmatter

\chapter{Introduction}

Number theory is a branch of pure mathematics dedicated to the study of integers. Regarded as one of the most fundamental areas of mathematics, it has deep connections with other disciplines such as analysis, algebra, and geometry. In modern times, number theory has found widespread applications, particularly in cryptography, where it plays a vital role in ensuring information security. Due to its profound theoretical significance and foundational importance, number theory is often hailed as the "Queen of Mathematics."

Elementary number theory is a fundamental branch of number theory that mainly studies integers and their basic properties using elementary methods. It traces back to ancient civilizations like the Egyptians and Babylonians who had basic number - related knowledge for practical needs. However, the ancient Greeks, particularly Pythagoras and his school, were among the first to study numbers for their own sake, exploring prime and perfect numbers. Euclid's "Elements" was a landmark, presenting the Euclidean algorithm and proving the infinity of prime numbers. During the Middle Ages, Arab mathematicians preserved and expanded on Greek works, while in Europe, Fibonacci introduced the Fibonacci sequence with number - theoretic implications. In the modern era, Fermat's theorems and conjectures, Euler's numerous contributions including the totient function, and Gauss's systematization in "Disquisitiones Arithmeticae" with concepts like congruences and quadratic reciprocity, significantly advanced the field. 

In this note, we use $\mathbb{Z}$, $\mathbb{N}$ and $\mathbb{Z}_{\geq 0}$ to denote the set of integers, positive integers and non-zero integers, respectively. We also call positive integers natural numbers and non-negative integers whole numbers.

\end{document}
